\documentclass[a4paper, fontsize=12pt, twoside]{article} % A4 paper and 11pt font size

\usepackage{xcolor}
%\pagecolor{black!100}
%\color{blue!60}

\usepackage[T1]{fontenc} % Use 8-bit encoding that has 256 glyphs
%\usepackage{fourier} % Use the Adobe Utopia font for the document - comment this line to return to the LaTeX default
\usepackage[english]{babel} % English language/hyphenation
\usepackage{amsmath,amsfonts,amsthm} % Math packages

\usepackage{sectsty} % Allows customizing section commands
\allsectionsfont{\centering \normalfont\scshape} % Make all sections centered, the default font and small caps

\usepackage{fancyhdr} % Custom headers and footers
\pagestyle{fancyplain} % Makes all pages in the document conform to the custom headers and footers	
\fancyhead{} % No page header - if you want one, create it in the same way as the footers below
\fancyfoot[L]{} % Empty left footer
\fancyfoot[C]{} % Empty center footer
\fancyfoot[R]{\thepage} % Page numbering for right footer
\renewcommand{\headrulewidth}{0pt} % Remove header underlines
\renewcommand{\footrulewidth}{0pt} % Remove footer underlines
\setlength{\headheight}{13.6pt} % Customize the height of the header



\setlength\parindent{0pt} % Removes all indentation from paragraphs - comment this line for an assignment with lots of text

%----------------------------------------------------------------------------------------
%	TITLE SECTION
%----------------------------------------------------------------------------------------





\newcommand{\Problem}[1]{{\large \underline{Problem #1}}}
\begin{document}
	\section{What is Git}
	\begin{enumerate}
		\item Used to tracking changes, especilly text.
		\item VCS: Version control system.
		\item SCM: Source code management.
		\item It is a distributed version control system:
			\begin{enumerate}
				\item Different users or teams of users maintain their own repository, instead of working from a central one.
				\item Changes are stored as ``changed sets'' or ``patches'', this tracks changes and not versions.
				\item No need to communicate with a central server.
				\item Faster.
				\item No network access required.
				\item No single failure point.
			\end{enumerate}
	\end{enumerate}
\section{Configuring Git}
\begin{enumerate}
	\item There are three places where git stores configuration information.
		\begin{enumerate}
			\item				System: /etc/gitconfig
			\item				User: ~/.gitconfig
			\item				Project: my\_project/.git/config
		\end{enumerate}
	\item Commands for each:
		\begin{enumerate}
			\item git config --system 
			\item git config --global
			\item git config
		\end{enumerate}
	\item To dislpay configurations use git config --list  OR u can specify which one.
\end{enumerate}
\section{Getting Started}
\begin{enumerate}
	\item git init //Initialize repository
	\item To stop VC then remove .git file.
	\item Usually only ever need to edit the config file in .git.
	\item It contains the project configurations.
	\item Basic git workflow:
		\begin{enumerate}
			\item make changes. //Add a new file
			\item add the changes. //git add .
			\item commit changes to the repository with message. //git commit -m "Initial commit"
		\end{enumerate}
	\item Write commit in the present tense.
	\item git log				
	\item git log --help
\end{enumerate}
\section{Git Concepts and Architecture}
\begin{enumerate}
	\item Git uses the three-tree architecture.
	\item This allows you to choose what you would like to commit.
	\item The HEAD pointer is a reference to the most recent commint of the checkedout branch.
	\item SHA is a unique 40 character hexadecimal string assigned to each commit.
\end{enumerate}
\section{Making Chnages to Files}
\begin{enumerate}
	\item git status //Shows difference between the working directory, staging index, and the repository.
	\item git diff //Compare changes for repository and staging index between working directory.
	\item git diff --staged //Compare changes between repository and staging index.
	\item git rm file\_name.
	\item git mv file newName //Renaming or moving are the same.
\end{enumerate}
\section{Using Git with a Rreal Project}
\begin{enumerate}
	\item git commit -am "Initial commit" //Does all in one go, caveat, if u delete a file or it is not tracked this will not commit it.
\end{enumerate}
	
\section{Remotes}
\begin{enumerate}
	\item You push your repository to the remote server.
	\item This create a new pointer to point at the latest commit, known as origin/master.
	\item You need to pull changes from other people by using a fetch.
	\item If your master is out of sync with the latest commint then need to merge.
	\item git remote gives list of all remotes it knows about.
	\item git remote add <name> <url>
	\item Can have as many remotes as you would like, usually first one is known as origin.
	\item git remote rm <name> //To remove remote.
		
	\item git clone <url> [newName]		
	\item Fetch syncronizes origin/master with remote repository.(Need Internet).
	\item git fetch //Need name if we have more than 1 repository.
	\item fetch only updates origin/master, not master.
	\item Tips:
		\begin{enumerate}
			\item fetch before you work.
			\item fetch before you push.
			\item fetch often.
			\item origin/master is just a remote brach, though we can not check it out.
			\item Once you fetch you need to merge.
			\item git pull = git fetch + git merge.
			\item git branch <name> <from>
			\item git branch -d <name> //This deletes branch <name>
		\end{enumerate}	
	\item git push origin :<branch> //Delete branch
	\item git push origin --delete <branch> //This is better
	\item git log p <branch>..origin/<branch>

	
\end{enumerate}
              
\end{document}

